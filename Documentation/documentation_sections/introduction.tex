\section{Introduction}
\label{sec:intro}
 
 Welcome to the DISCO software: a software for segmenting yeast cells in trap like microfluidic devices, tracking them and viewing/editing the results. First, a few notes:
 \begin{itemize}
 	\item The software requires that the swain lab general functions package also be installed, which can be found at \href{https://github.com/pswain/GeneralMatlabFunctions}{https://github.com/pswain/GeneralMatlabFunctions}.
 	\item The best description of the underlying algorithm is the associated paper published at\\ \href{https://academic.oup.com/bioinformatics/article-abstract/doi/10.1093/bioinformatics/btx550/4103414?redirectedFrom=fulltext}{https://academic.oup.com/bioinformatics/article-abstract/doi/10.1093/bioinformatics/btx550/4103414?redirectedFrom=fulltext}.
 	\item The software was developed in the swain lab, and you should feel free to contact them with help and support in using the software (\href{http://swainlab.bio.ed.ac.uk/}{http://swainlab.bio.ed.ac.uk/}).
 	\item It is recommended that you use the software with MATLAB 2015b. Using it with later versions of matlab requires some extra setup steps that are detailed in section \ref{sec:other_matlabs}.
 \end{itemize}
 
 There are broadly two parts to the software: using the software to segment images and training the software to segment unseen image types. The first, segmenting, is relatively straightforward and is how you will use the software day to day. The second, training, requires a little more expertise and effort but should only need to be done once for a given imaging modality/microscope. \\
 We will first describe segmentation, and then describe training, even though it may be necessary for somebody in your group to train a model before anyone is able to segment images.